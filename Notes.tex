\documentclass[11pt,a4paper]{article}

\usepackage[utf8]{inputenc}
\usepackage{parskip}
\usepackage{tabularx}
\usepackage{amsmath}
\usepackage{amssymb}
\usepackage{geometry}
\usepackage{booktabs}
\usepackage{centernot}
\usepackage{hyperref}
\usepackage{eufrak}

\geometry{a4paper, left=20mm, right=20mm, top=20mm, bottom=20mm}

\usepackage{fancyhdr}
\pagestyle{fancy}
\lhead{Anthony Catterwell}
\chead{\textsc{University of Edinburgh}}
\rhead{Honours Algebra}

\title{Honours Algebra Notes}
\author{Anthony Catterwell}

\begin{document}
\maketitle
\tableofcontents

\break\

\section{Vector Spaces}

\subsection{Solutions of simultaneous linear equations}

\begin{itemize}

    \item \textbf{Theorem 1.1.4} \emph{Solution sets of inhomogeneous systems of linear
        equations} \\
        If the solution set of a linear system of equations is non-empty,
        then we obtain all solutions by adding component-wise an arbitrary solution
        of the associated homogenised system to a fixed solution of the system.
\end{itemize}

\subsection{Fields \& vector spaces}

\begin{itemize}

    \item \textbf{Definition 1.2.1.1} \emph{Fields} \\
        A \emph{field} $F$ is a set with functions
        \begin{align*}{}
            \text{addition} \  &= + : F \times F \to F \ ;
            \ (\lambda, \mu) \mapsto \lambda + \mu \\
            \text{multiplication} \ &=. : F \times F \to F ; (\lambda, \mu) \mapsto \lambda\mu
        \end{align*}
        such that $(F, +)$ and $(F \ \setminus \{0\},.)$ are abelian groups, with
        \[
            \lambda (\mu + \nu) = \lambda \mu + \lambda \nu \in F, \quad
            \forall \lambda \nu \in F
        \]
        The neutral elements are called $0_F, 1_F$.
        In particular
        \[
            \lambda + \mu = \mu + \lambda ,\
            \lambda. \mu  = \mu. \lambda ,\
            \lambda + 0_F = \lambda ,\
            \lambda. 1_F  = \lambda \in F, \quad
            \forall \lambda, \mu \in F
        \]
        For every $\lambda \in F$ there exists $-\lambda \in F$ such that
        \[
            \lambda + (-\lmbda) = 0_F \in F
        \]
        For every $\lambda \neq 0 \in F$ there exists $\lambda^{-1} \neq 0 \in F$ such that
        \[
            \lambda(\lambda^{-1}) = 1_F \in F
        \]

    \item \textbf{Definition 1.2.1.2} \emph{Vector space} \\
        A \emph{vector space} $V$ over a \emph{field} $F$
        is a pair consisting of an abelian group $V = (V,+)$ and a mapping
        \[
            F \times V \to V : (\lambda, \textbf{v}) \mapsto \lambda \textbf{v}
        \]
        such that for all $\lambda, \mu \in F$ and $\textbf{v}, \textbf{w} \in V$
        the following identities hold:
        \begin{align*}{}
            \lambda(\textbf{v} + \textbf{w}) & = (\lambda \textbf{v}) + (\lambda \textbf{w})
            \quad                            & \text{(distributivity)} \\
            (\lambda + \mu) \textbf{v}       & = (\lambda \textbf{v}) + (\mu \textbf{v})
            \quad                            & \text{(distributivity)} \\
            \lambda(\mu \textbf{v})          & = (\lambda \mu) \textbf{v}
            \quad                            & \text{(associativity)} \\
            1_F\textbf{v}                    & = \textbf{v}
        \end{align*}
        A vector space $V$ over a field $F$ is called an $F$-\emph{vector space}.

    \item \textbf{Lemma 1.2.2} Product with the scalar zero \\
        If $V$ is a vector space and $\textbf{v} \in V$, then $0\textbf{v} = \textbf{0}$

    \item \textbf{Lemma 1.2.3} Product with the scalar $(-1)$ \\
        If $V$ is a vector space and $\textbf{v} \in V$, then $(-1)\textbf{v} = -\textbf{v}$.

    \item \textbf{Lemma 1.2.4} Product with the zero vector \\
        If $V$ is a vector space over a field $F$, then $\lambda\textbf{0} = \textbf{0}$
        for all $\lambda \in F$.
        Furthermore, if $\lambda \textbf{v} = \textbf{0}$,
        then either $\lambda = 0$ or $\textbf{v} = \textbf{0}$.
\end{itemize}

\subsection{Products of sets and of vector spaces}

\subsection{Vector subspaces}

\begin{itemize}

    \item \textbf{Definition 1.4.1} \emph{Vector subspaces} \\
        A subset $U$ of a vector space $V$ is called a \emph{vector subspace} or \emph{subspace}
        if $U$ contains $\textbf{0}$ and
        \[
            \textbf{u}, \textbf{v} \in U  \text{and} \ \lambda \in F \implies
            \textbf{u} + \textbf{v} \in U \ \text{and} \ \lambda \textbf{u} \in U
        \]

    \item \textbf{Proposition 1.4.5} Generating a vector subspace from a subset \\
        Let $T$ be a subset of a vector space $V$ over a field $F$.
        Then amongst all vector subspace of $V$ that include $T$,
        there is a smallest vector subspace
        \[
            \langle T \rangle = \langle T \rangle _F \subseteq V
        \]
        It can be described as the set of all vectors
        $\alpha_1 \textbf{v}_1 + \cdots + \alpha_r \textbf{v}_r$ with
        $\alpha_1, \ldots, \alpha_r \in F$ and $\textbf{v}_1, \ldots, \textbf{v}_r \in T$,
        together with $\textbf{0}$ in the case $T = \emptyset$.

    \item \textbf{Definition 1.4.7} \emph{Generating set} \\
        A subset of a vector space is called a \emph{generating set} of our vector space if its
        span is all of the vector space.
        A vector space that has a finite generating set is said to be \emph{finitely generated}.

    \item \textbf{Definition 1.4.9} \\
        The set of all subsets $\mathcal{P}(X) = \{U : U \subseteq X \}$ of $X$ is the
        \emph{power set} of $X$. \\
        A subset of $\mathcal{P}(X)$ is a \emph{system of subsets of} $X$. \\
        Given such a system $\mathcal{U} \subseteq \mathcal{P}(X)$ we can create two new subsets
        of $X$,
        the \emph{union} and the \emph{intersection} of the sets of our system $\mathcal{U}$:
        \begin{align*}{}
            \bigcup_{U \in \mathcal{U}} U &=
            \{x \in X : \exists U \in \mathcal{U} \ldotp x \in U\} \\
            \bigcap_{U \in \mathcal{U}} U &= \{x \in X : x \in U \ \forall \ U \in \mathcal{U}\}
        \end{align*}
        In particular the intersection of the empty system of subsets of $X$ is $X$,
        and the union of the empty system of subsets $X$ is the empty set.
\end{itemize}

\subsection{Linear independence and bases}

\begin{itemize}

    \item \textbf{Definition 1.5.1} \emph{Linear independence} \\
        A subset $L$ of a vector space $V$ is \emph{linearly independent}
        if for all pairwise different vectors
        $\textbf{v}_1, \ldots, \textbf{v}_r \in L$ and arbitrary vectors
        $\alpha_1, \ldots, \textbf{v}_r \in F$,
        \[
            \alpha_1 \textbf{v}_1 + \cdots + \alpha_r \textbf{v}_r = \textbf{0} \implies
            \alpha_1 = \cdots = \alpha_r = 0
        \]

    \item \textbf{Definition 1.5.2} \emph{Linear dependence} \\
        A subset $L$ of a vector space $V$ is called \emph{linearly dependent} if it is not
        linearly independent.

    \item \textbf{Definition 1.5.8} \emph{Basis} \\
        A \emph{basis} of a vector space $V$ is a linearly independent generating set in $V$.

    \item \textbf{Theorem 1.5.11} Linear combinations of basis elements \\
        Let $F$ be a field, $V$ be a vector space over $F$, and
        $\textbf{v}_1, \ldots, \textbf{v}_r \in V$ vectors.
        The family ${(\textbf{v}_i)}_{1 \leq i \leq r}$ is a basis of $V$ if and only if the
        following ``evaluation'' mapping
        \begin{align*}{}
            \Phi : F^r                    & \to V \\
            (\alpha_1, \ldots, \alpha_r)  & \mapsto \alpha\textbf{v}_1 +
            \cdots + \alpha_r\textbf{v}_r
        \end{align*}
        is a bijection.

    \item \textbf{Theorem 1.5.12} Characterisation of bases \\
        The following are equivalent for a subset $E$ of a vector space $V$:
        \begin{enumerate}
            \item $E$ is a basis, i.e.\ a linearly independent generating set;
            \item $E$ is minimal among all generating sets,
                meaning that $E \ \setminus \{\textbf{v}\}$ does not generate $V$,
                $\forall \textbf{v} \in E$;
            \item $E$ is maximal among all linearly independent subsets,
                meaning that $E \cup \{\textbf{v}\}$ is not linearly independent
                $\forall \textbf{v} \in V$.
        \end{enumerate}

    \item \textbf{Corollary 1.5.13} The existence of a basis \\
        Let $V$ be a finitely generated vector space over a field $F$.  The $V$ has a basis.

    \item \textbf{Theorem 1.5.14} (Useful variant on the Characterisation of bases) \\
        Let $V$ be a vector space.
        \begin{enumerate}
            \item If $L \subset V$ is a linearly independent subset and $E$ is minimal
                amongst all generating sets of our vector space with the property that
                $L \subseteq E$, then $E$ is a basis.
            \item If $E \subseteq V$ is a generating set and if $L$ is maximal amongst all
                linearly independent subsets of our vector space with the property
                $L \subseteq E$, then $L$ is  basis.
        \end{enumerate}

    \item \textbf{Definition 1.5.15} \\
        Let $X$ be a set and $F$ a field.
        The set $\text{Maps}(X,F)$ of all mappings $f : X \to F$ becomes an $F$-vector space
        with the operations of point-wise addition and multiplication by a scalar.
        The subset of all mappings which send almost all elements of
        $X$ to zero is a vector subspace
        \[
            F \langle X \rangle \subseteq \text{Maps}(X,F)
        \]
        This vector subspace is called the \emph{free vector space on the set} $X$.

    \item \textbf{Theorem 1.5.16} (Useful variant on Linear combinations of basis elements) \\
        Let $F$ be a field, $V$ an $F$-vector space, and ${(\textbf{v}_i)}_{i\in I}$
        a family of vectors from the vector space $V$.
        The following are equivalent:
        \begin{enumerate}
            \item The family $(\textbf{v}_i){i\in I}$ is a basis for $V$;
            \item For each vector $\textbf{v} \in V$ there is precisely one family
                ${(a_i)}_{i \in I}$ of elements of our field $F$,
                almost all of which are zero and such that
                \[
                    \textbf{v} = \sum_{i \in I} a_i \textbf{v}_i
                \]
        \end{enumerate}
\end{itemize}

\subsection{Dimension of a vector space}

\begin{itemize}

    \item \textbf{Theorem 1.6.1} Fundamental estimate of linear algebra \\
        No linearly independent subset of a given vector space has more elements than a
        generating set.
        Thus if $V$ is a vector space, $L \subset V$ a linearly independent subset,
        and $E \subseteq V$ a generating set, then:
        \[
            |L| \leq |E|
        \]

    \item \textbf{Theorem 1.6.2} Steinitz exchange theorem \\
        Let $V$ be a vector space, $L \subset V$ and finite linearly independent subset,
        and $E \subseteq V$ and generating set.
        Then there is an injection $\Phi : L \to E$ such that
        $(E \ \setminus \Phi(L)) \cup L$ is also a generating set for $V$.

        We can swap out some elements of a generating set by the elements of our linearly
        independent set, and still keep a generating set.

    \item \textbf{Lemma 1.6.3} Exchange lemma \\
        Let $V$ be a vector space, $M \subseteq V$ a linearly independent subset,
        and $E \subseteq V$ a generating subset, such that $M \subseteq E$.
        If $\textbf{w} \in V \ \setminus M$ is a vector set not belonging to $M$ such that
        $M \cup \{\textbf{w}\}$ is linearly independent, then there exists
        $\textbf{e} \in E \ \setminus M$ such that
        $\{E \ \setminus \{\textbf{e}\}\} \cup \{\textbf{w} \}$ is a generating set for $V$.

    \item \textbf{Corollary 1.6.4} Cardinality of bases \\
        Let $V$ be a finitely generated vector space.
        \begin{enumerate}
            \item $V$ has a finite basis;
            \item $V$ cannot have an infinite basis;
            \item Any two bases of $V$ have the same number of elements.
        \end{enumerate}

    \item \textbf{Definition 1.6.5} \emph{Dimension} \\
        The cardinality of one (and each) basis of a finitely generated vector space $V$
        is called the \emph{dimension} of $V$ and is denoted $\text{dim}V$.
        If the vector space is not finitely generated, then $\text{dim}V = \infty$
        and $V$ is \emph{infinite dimensional}.

    \item \textbf{Corollary 1.6.8} Cardinality criterion for bases \\
        Let $V$ be a finitely generated vector space.
        \begin{enumerate}
            \item Each linearly independent subset $L \subset V$ has at most dim$V$ elements,
                and if $|L| = \text{dim}V$, then $L$ is actually a basis;
            \item Each generating set $E \subseteq V$ has at least dim$V$ elements,
                and if $|E| = \text{dim}V$ then $E$ is actually a basis.
        \end{enumerate}

    \item \textbf{Corollary 1.6.9} Dimension estimate for vector subspaces \\
        A proper vector subspace of a finite dimensional vector space has itself a strictly
        smaller dimension.

    \item \textbf{Theorem 1.6.11} The dimension theorem \\
        Let $V$ be a vector space containing vector subspaces $U, W \subseteq V$.  Then
        \[
            \text{dim}(U+W) + \text{dim}(U \cap W) = \text{dim}U + \text{dim}W
        \]
\end{itemize}

\subsection{Linear mappings}

\begin{itemize}

    \item \textbf{Definition 1.7.1} \emph{Linear mapping} \\
        Let $V,W$ be vector spaces over a field $F$.
        A mapping $f: V \to W$ is called \emph{linear}
        if for all $\textbf{v}_1, \textbf{v}_2 \in V$ and $\lambda \in F$ we have
        \begin{align*}{}
            f(\textbf{v}_1 + \textbf{v}_2) & = f(\textbf{v}_1) + f(\textbf{v}_2) \\
            f(\lambda\textbf{v}_1)         & = \lambda f(\textbf{v}_1)
        \end{align*}
        A bijective linear mapping is called an \emph{isomorphism} of vector spaces.
        If there is an isomorphism of vector spaces, we call them \emph{isomorphic}.
        A homomorphism from one vector space to itself is called an \emph{endomorphism}.
        An isomorphism of a vector space to itself is called an \emph{automorphism}.

    \item \textbf{Definition 1.7.5} \emph{Fixed point} \\
        A point that is sent to itself by a mapping is called a \emph{fixed point} of the mapping.
        Given a mapping $f: X \to X$, we denote the set of fixed points by
        \[
            X^f = \{x \in X : f(x) = x\}
        \]

    \item \textbf{Definition 1.7.6} \emph{Complementary} \\
        Two vector subspace $V_1, V_2$ of a vector space $V$ are \emph{complementary} if addition
        defines a bijection
        \[
            V_1 \times V_2 \to V
        \]

    \item \textbf{Theorem 1.7.7} Classification of vector spaces by their dimension \\
        Let $n \in \mathbb{N}$.
        Then a vector space over a field $F$ is isomorphic to $F^n$ if and only if it has
        dimension $n$.

    \item \textbf{Lemma 1.7.8} Linear mappings and bases \\
        Let $V,W$ be vector spaces over $F$ and let $B \subset V$ be a basis.
        Then restriction of a mapping gives a bijection
        \begin{align*}{}
            \text{Hom}_F(V,W) & = \text{Hom}(V,W) \subseteq \text{Maps}(V,W) \\
            f                 & \mapsto f|_B
        \end{align*}
        In other words, each linear mapping determines and is completely determined by the values
        it takes on a basis.

    \item \textbf{Proposition 1.7.9}
        \begin{enumerate}
            \item Every injective linear mapping $f : V \to W$ has a \emph{left inverse},
                in other words a linear mapping $g : W \to V$ such that $g \circ f = \text{id}_V$
            \item Every surjective linear mapping $f: V \to W$ has a \emph{right inverse},
                in other words a linear mapping $g : W \to V$ such that $f \circ g = \text{id}_W$
        \end{enumerate}
\end{itemize}

\subsection{Rank-Nullity theorem}

\begin{itemize}

    \item \textbf{Definition 1.8.1} \\
        The \emph{image} of a linear mapping $f : V \to W$ is the subset
        $\text{im}(f) = f(V) \subseteq W$.
        It is a vector subspace of $W$.
        The pre-image of the zero vector of a linear mapping $f : V \to W$ is denoted by
        \[
            \text{ker}(f) \equiv f^{-1}(0) = \{v \in V : f(v) = 0 \}
        \]
        and is called the \emph{kernel} of the linear mapping $f$.
        The kernel is a vector subspace of $V$.

    \item \textbf{Lemma 1.8.2} \\
        A linear mapping $f : V \to W$ is injective if and only if $\text{ker}_f = 0$.

    \item \textbf{Theorem 1.8.4} Rank-Nullity theorem \\
        Let $f : V \to W$ be a linear mapping between vector spaces. Then
        \begin{align*}{}
            \text{dim}V &= \text{dim}(\text{ker}f) + \text{dim}(\text{im}f) \\
                        &= \text{nullity} \ + \ \text{rank}
        \end{align*}

    \item \textbf{Corollary 1.8.5} (Dimension theorem, again) \\
        Let $V$ be a vector space, and $U,W \subseteq V$ vector subspaces. Then
        \[
            \text{dim}(U + W) + \text{dim}(U \cap W) = \text{dim}U + \text{dim}W
        \]
\end{itemize}

\section{Linear Mappings and Matrices}

\subsection{Linear mappings $F^m \to F^n$ and matrices}

\begin{itemize}

    \item \textbf{Theorem 2.1.1} Linear mappings $F^m \to F^n$ and matrices \\
        Let $F$ be a field and let $m, n \in \mathbb{N}$.
        There is a bijection between the space of linear mappings $F^m \to F^n$
        and the set of matrices with $n$ rows and $m$ columns and entries in $F$
        \begin{align*}{}
            \mathrm{M} : \text{Hom}_F(F^m, F^n) &\to \ \text{Mat}(n \times m ; F) \\
            f &\mapsto [f]
        \end{align*}
        This attaches to each linear mapping $f$ its \emph{representing matrix}
        $\mathrm{M}(f) \equiv [f]$.
        The columns of this matrix are the images under $f$ of the standard basis elements of $F^m$
        \[
            [f] \equiv (f(\textbf{e}_1)|f(\textbf{e}_2)| \cdots | f(\textbf{e}_m))
        \]

    \item \textbf{Definition 2.1.6} \emph{Product} \\
        Let $n, m, l \in \mathbb{N}$, $F$ and field, and let
        $A \in \mathrm{Mat}(n \times m; F)$ and $B \in \mathrm{Mat}(m \times l; F)$
        be matrices.
        The \emph{product} $A \circ B = AB \in \mathrm{Mat}(n \times l; F)$
        is the matrix defined by
        \[
            {(AB)}_{ik} = \sum_{j=1}^m A_{ij}B_{jk}
        \]
        Matrix multiplication produces a mapping
        \begin{align*}{}
            \mathrm{Mat}(n \times m;F) \times \mathrm{Mat}(m \times l; F) &\to
            \mathrm{Mat}(m \times l; F) \\
            (A,B) &\mapsto AB
        \end{align*}

    \item \textbf{Theorem 2.1.8} Composition of linear mappings and products of matrices \\
        Let $g : F^l \to F^m$ and $f : F^m \to F^n$ be linear mappings.
        The representing matrix of their composition is the product of their representing matrices
        \[
            [f \circ g] = [f] \circ [g]
        \]

    \item \textbf{Proposition 2.1.9} Calculating with matrices \\
        Let $k, l, m, n \in \mathbb{N}, A, A' \in \mathrm{Mat}(n \times m;F),
        B, B' \in \mathrm{Mat}(m \times l;F), C \in \mathrm{Mat}(l \times k; F)$ and $I = I_m$.
        Then the following hold for matrix multiplication
        \begin{align*}{}
            (A + A')B & = AB + A'B \\
            A(B + B') & = AB + AB' \\
            IB        & = B \\
            AI        & = A \\
            (AB)C     & = A(BC)
        \end{align*}
\end{itemize}

\subsection{Basic properties of matrices}

\begin{itemize}

    \item \textbf{Definition 2.2.1} \emph{Invertible} \\
        A matrix $A$ is called \emph{invertible} if there exist matrices $B$ and $C$ such that
        $BA = I$ and $AC = I$.

    \item \textbf{Definition 2.2.2} \emph{Elementary matrix} \\
        An \emph{elementary matrix} is any square matrix that differs from the identity matrix
        in at most one entry.

    \item \textbf{Theorem 2.2.3} \\
        Every square matrix can be written as a product of elementary matrices.

    \item \textbf{Definition 2.2.4} \emph{Smith Normal Form} \\
        Any matrix whose only non-zero entries lie on the diagonal,
        and which has first 1s on along the diagonal followed by 0s is in \emph{Smith Normal Form}.

    \item \textbf{Theorem 2.2.5} Transformation of a matrix into Smith-Normal form \\
        For each matrix $A \in \mathrm{Mat}(n \times m; F)$ there exist invertible matrices
        $P$ and $Q$ such that $PAQ$ is a matrix in Smith Normal Form and $Q$ such that $PAQ$ is a
        matrix in Smith Normal Form.

    \item \textbf{Definition 2.2.6} \emph{Rank} \\
        The \emph{column rank} of a matrix $A \in \mathrm{Mat}(n \times m; F)$
        is the dimension of the subspace of $F^n$ generated by the columns of $A$.
        Similarly, the \emph{row rank} of $A$ is the dimension of the subspace of $F^m$ generated
        by the rows of $A$.

    \item \textbf{Theorem 2.2.7} \\
        The column rank and the row rank of any matrix are equal.

    \item \textbf{Definition 2.2.8} \emph{Full rank} \\
        Whenever the rank of a matrix is equal to the number of rows (or columns --- whichever is
        smaller), it has \emph{full rank}.
\end{itemize}

\subsection{Abstract linear mappings and matrices}

\begin{itemize}

    \item \textbf{Theorem 2.3.1} Abstract linear mappings and matrices \\
        Let $F$ be a field, $V$ and $W$ vector spaces over $F$ with ordered bases
        $\mathcal{A} = (\textbf{v}_1, \ldots, \textbf{v}_m)$ and
        $\mathcal{B} = (\textbf{w}_1, \ldots, \textbf{w}_n)$.
        Then to each linear mapping $f : V \to W$ we associated a \emph{representing matrix}
        $_\mathcal{B}{[f]}_\mathcal{A}$ whose entries $a_{ij}$ are defined by the identity
        \[
            f(\textbf{v}_j) = a_{1j}\textbf{w}_1 + \cdots + a_{nj}\textbf{w}_n \in W
        \]
        This produces a bijection, which is even an isomorphism of vector spaces
        \begin{align*}{}
            \mathrm{M}_\mathcal{B}^\mathcal{A} : \mathrm{Hom}_F(V,W) &\to
            \mathrm{Mat}(n \times m; F) \\
            f &\mapsto _\mathcal{B}{[f]}_\mathcal{A}
        \end{align*}

    \item \textbf{Theorem 2.3.2} The representing matrix of a composition of linear mappings \\
        Let $F$ be a field and $U, V, W$ finite-dimensional vector spaces over $F$ with ordered
        bases $\mathcal{A, B, C}$
        If $f : U \to V$ and $g : V \to W$ are linear mappings,
        then the representing  matrix of the composition
        $g \circ f : U \to W$
        is the matrix product of the representing matrices of $f$ and $g$
        \[
            _\mathcal{C}{[g \circ f]}_\mathcal{A} = _\mathcal{C}{[g]}_\mathcal{B} \circ
            _\mathcal{B}{[f]}_\mathcal{A}
        \]

    \item \textbf{Definition 2.3.3} \\
        Let $V$ be a finite-dimensional vector spaces with an ordered basis
        $\mathcal{A} = (\textbf{v}_1, \ldots, \textbf{v}_m)$
        We denote the inverse to the bijection
        $\Phi_\mathcal{A} : F^m \to V, {(\alpha_1, \ldots, \alpha_m)}^T \mapsto
        \alpha_1\textbf{v}_1 + \cdots + \alpha_m\textbf{v}m$ by
        \[
            \textbf{v} \mapsto _\mathcal{A}[\textbf{v}]
        \]
        The column vector $_\mathcal{A}[\textbf{v}]$ is called the \emph{representation of the
        vector $\textbf{v}$ with respect to the basis $\mathcal{A}$}.

    \item \textbf{Theorem 2.3.4} Representation of the image of a vector \\
        Let $V,W$ be finite-dimensional vector-spaces over $F$ with ordered bases $\mathcal{A,B}$
        and let $f : V \to W$ be a linear mapping.
        The following holds for $\textbf{v} \in V$:
        \[
            _\mathcal{B}{[f(\textbf{v})]} = _\mathcal{B}{[f]}_\mathcal{A} \circ
            _\mathcal{A}[\textbf{v}]
        \]
\end{itemize}

\subsection{Change of a matrix by change of basis}

\begin{itemize}

    \item \textbf{Definition 2.4.1} \emph{Change of basis matrix} \\
        Let $\mathcal{A} = (\textbf{v}_1, \ldots, \textbf{v}_n)$ and $\mathcal{B} =
        (\textbf{w}_1, \ldots, \textbf{w}_n)$
        be ordered bases of the same $F$-vector space $V$.
        Then the matrix representing the identity mapping with respect to these bases
        \[
            _\mathcal{B}{[\mathrm{id}_V]}_\mathcal{A}
        \]
        is called a \emph{change of basis matrix}.
        By definition, its entries are given by the equalities $\textbf{v}_j =
        \sum_{i=1}^n a_{ij}\textbf{w}_i$.

    \item \textbf{Theorem 2.4.3} Change of basis \\
        Let $V$ and $W$ be finite-dimensional vector-spaces over $F$ and let $f : V \to W$
        be a linear mapping.
        Suppose that $\mathcal{A, A'}$ are ordered bases of $V$ and $\mathcal{B, B'}$
        are ordered bases of $W$.
        Then
        \[
            _\mathcal{B'}{[f]}_\mathcal{A'} = _\mathcal{B'}{[\mathrm{id}_W]}_\mathcal{B} \circ
            _\mathcal{B}{[\mathrm{f}]}_\mathcal{A} \circ _\mathcal{A}{[\mathrm{id}_V]}_\mathcal{A'}
        \]

    \item \textbf{Corollary 2.4.4}
        Let $V$ be a finite-dimensional vector-space and let
        $f : V \to V$ be an endomorphism of $V$.
        Suppose that $\mathcal{A, A'}$ are ordered bases of $V$.
        Then
        \[
            _\mathcal{A'}{[f]}_\mathcal{A'} = _\mathcal{A'}{[\mathrm{id}_V]}^{-1}_\mathcal{A'}
            \circ _\mathcal{A}{[\mathrm{f}]}_\mathcal{A} \circ _\mathcal{A}
            {[\mathrm{id}_V]}_\mathcal{A'}
        \]

    \item \textbf{Theorem 2.4.5} Smith Normal Form \\
        Let $f : V \to W$ be a linear mapping between finite-dimensional $F$-vector spaces.
        There exist an ordered basis $\mathcal{A}$ of $V$ and an ordered basis
        $\mathcal{B}$W of $W$
        such that the representing matrix $_\mathcal{B}{[f]}_\mathcal{A}$
        has zero entries everywhere except possibly on the diagonal,
        and along the diagonal there are 1s first, followed by 0s.

    \item \textbf{Definition 2.4.6} \emph{Trace} \\
        The \emph{trace} of a square matrix is defined to be the sum of its diagonal entries.
        We denote this by
        \[
            \mathrm{tr}(A)
        \]
\end{itemize}

\section{Rings and Modules}

\subsection{Rings}

\begin{itemize}

    \item \textbf{Group Axioms}
        \begin{enumerate}
            \item Closure
            \item Associativity
            \item Existence of identity
            \item Existence of inverses
        \end{enumerate}

    \item \textbf{Definition 3.3.1} \emph{Ring} \\
        A \emph{ring} is a set with two operations $(R,+,.)$ that satisfy
        \begin{enumerate}
            \item $(R,+)$ is an abelian group;
            \item $(R, \cdot)$ is a \emph{monoid}; this means that the second operation
                $\cdot : R \cdot R \to R$ is associative and that there is an
                \emph{identity element} $1=1_R \in R$.
            \item The distributive laws hold.
        \end{enumerate}
        The two operations are called \emph{addition} and \emph{multiplication} in our ring. \\
        A ring in which multiplication is commutative is a \emph{commutative ring}. \\

    \item \textbf{Proposition 3.1.7} Divisibility by sum \\
        A natural number is divisible by 3 (respectively 9) precisely when the sum of its digits is
        divisible by 3 (respectively 9).

    \item \textbf{Definition 3.1.8} \emph{Field} \\
        A \emph{field} $F$ is a non-zero commutative ring in which every non-zero element
        $a \in F$ has an inverse $a^{-1} \in F$.

    \item \textbf{Proposition 3.1.11} \\
        Let $m \in \mathbb{Z}^+$.
        The commutative ring $\mathbb{Z} / m\mathbb{Z}$ is a field if and only if $m$ is prime.
\end{itemize}

\subsection{Properties of rings}

\begin{itemize}

    \item \textbf{Lemme 3.2.1} Additive inverses \\
        Let $R$ be a ring and let $a, b \in R$.
        Then
        \begin{enumerate}
            \item $0a = 0 = a0$
            \item $(-a)b = -(ab) = a(-b)$
            \item $(-a)(-b) = ab$
        \end{enumerate}

    \item \textbf{Definition 3.2.3} \\
        Let $m \in \mathbb{Z}$.
        The \emph{$m$-th multiple $ma$ of an element} a in abelian group $R$ is
        \[
            ma = \underbrace{a + a + \cdots + a}_{m \ \text{terms}} \quad \text{ if } m > 0
        \]
        $0a = 0$, and negative multiples are defined by $(-m)a = -(ma)$.

    \item \textbf{Lemma 3.2.4} Rules for multiples \\
        Let $R$ be a ring, let $a,b \in R$ and let $m,n \in \mathbb{Z}$.
        Then
        \begin{enumerate}
            \item $m(a+b)   = ma + mb$;
            \item $(m+n)a   = ma + na$;
            \item $m(na)    = (mn)a$;
            \item $m(ab)    = (ma)b      = a(mb)$;
            \item $(ma)(nb) = (mn)(ab)$;
        \end{enumerate}

    \item \textbf{Definition 3.2.6} \emph{Unit} \\
        Let $R$ be a ring.
        An element $a \in R$ is called a \emph{unit} if it is invertible in $R$ or (in other words)
        has a multiplicative inverse in $R$.

    \item \textbf{Proposition 3.2.10} \\
        The set $R^\times$ of units in a ring $R$ forms a group under multiplication.

    \item \textbf{Definition 3.2.13} \emph{Integral domains} \\
        An \emph{integral domain} is a non-zero commutative ring that has no zero-divisors.

    \item \textbf{Proposition 3.2.16} Cancellation law for integral domains \\
        Let $R$ be an integral domain and let $a,b,c \in R$.
        \[
            ab = ac \ \text{and} \ a \neq 0 \implies b = c
        \]

    \item \textbf{Proposition 3.2.17} \\
        Let $m \in \mathbb{N}$.
        Then $\mathbb{Z}/m\mathbb{Z}$ is an integral domain if and only if $m$ is prime.

    \item \textbf{Theorem 3.2.18} \\
        Every \emph{finite} integral domain is a field.
\end{itemize}

\subsection{Polynomials}

\begin{itemize}

    \item \textbf{Definition 3.1.1} \\
        Let $R$ be a ring.
        A \emph{polynomial over} $R$ is an expression of the form
        \[
            P = a_0 + a_1X + a_2{X^2} + \cdots + a_m{X^m}
        \]
        for some $m \in \mathbb{N}$ and elements $a_i \in R$ for $i \in [0,m]$.\\
        The set of all polynomials over $R$ is denoted by $R[X]$.\\
        In case $a_m$ is non-zero, the polynomial $P$ has \emph{degree} $m$, written $\deg(P)$, and
        $a_m$ is its \emph{leading coefficient}. \\
        When the leading coefficient is 1, the polynomial is a \emph{monic polynomial}.\\
        A polynomial of degree one is called \emph{linear},
        a polynomial of degree two is called \emph{quadratic},
        and a polynomial of degree three is called \emph{cubic}.

    \item \textbf{Definition 3.3.2} \emph{Ring of polynomials} \\
        The set $R[X]$ is a ring called the \emph{ring of polynomials over $R$}.
        The zero and the identity of $R[X]$ are the zero and identity of $R$, respectively.

    \item \textbf{Lemma 3.3.3}
        \begin{enumerate}
            \item If $R$ is ring with no zero-divisors, then $R[X]$ has no zero-divisors and
                $\deg(PQ) = \deg(P) + \deg(Q)$ for non-zero $P,Q \in R[X]$.
            \item If $R$ is an integral domain, then so is $R{[X]}$
        \end{enumerate}

    \item \textbf{Theorem 3.3.4} Division and remainder \\
        Let $R$ be an integral domain, and let $P,Q \in R[X]$ with $Q$ monic.
        Then there exists unique $A,B \in R{[X]}$ such that
        $P=AQ + B$ and $\deg(B) < \deg(Q)$ or $B=0$.

    \item \textbf{Definition 3.3.6} \\
        Let $R$ be a commutative ring and $P \in R[X]$ a polynomial.
        Then the polynomial $P$ can be \emph{evaluated} at $\lambda \in R$ to produce $P(\lambda)$
        by replacing the powers of $X$ in the polynomial $P$ by the corresponding powers of
        $\lambda$.
        This gives a mapping
        \[
            R[X] \to \mathrm{Maps}(R,R)
        \]
        An element $\lambda \in R$ is a \emph{root} of $P$ if $P(\lambda) = 0$.

    \item \textbf{Proposition 3.3.9} \\
        Let $R$ be a commutative ring, let $\lambda \in R$ and $P(X) \in R[X]$.
        Then $\lambda$ is a root of $P(X)$ if and only if $(X-\lambda)$ divides $P(X)$.

    \item \textbf{Theorem 3.3.10} \\
        Let $R$ a ring, or more generally, an integral domain.
        Then an non-zero polynomial $P \in R[X] \ \setminus \{0\}$ has at most $\deg(P)$ roots in
        $R$.

    \item \textbf{Definition 3.3.11} \emph{Algebraically closed} \\
        A field $F$ is \emph{algebraically closed} if each non-constant polynomial
        $P \in F[X] \ \setminus F$ with coefficients $F$ has a root in $F$.

    \item \textbf{Theorem 3.3.13} \emph{Fundamental theorem of algebra} \\
        If $F$ is an algebraically closed field, then every non-zero polynomial
        $P \in F[X] \ \setminus \{0\}$ \emph{decomposes into linear factors}
        \[
            P = c(X - \lambda_1) \cdots (X - \lambda_n)
        \]
        with $n \geq 0, c \in F^\times$ and $\lambda_1, \ldots, \lambda_n \in F$.
        This decomposition is unique up to reordering of the factors.
\end{itemize}

\subsection{Homomorphisms, Ideals, and Subrings}

\begin{itemize}

    \item \textbf{Definition 3.4.1} \emph{Ring homomorphism} \\
        Let $R$ and $S$ be rings.
        A mapping $f : R \to S$ is a \emph{ring homomorphism} if the following hold
        $\forall x,y\in R$
        \begin{align*}{}
            f(x+y) & = f(x) + f(y) \\
            f(xy)  & = f(x)f(y)
        \end{align*}

    \item Prelude to ideals \\
        Let $f : R \to S$ be a ring homomorphism with $\ker f = \{ r \in R : f(r) = 0_S \}$.
        Then $\ker f$ is:
        \begin{itemize}
            \item a subgroup of $R$ under addition
            \item $0_R \in \ker f$
            \item closed under multiplication
            \item closed under left and right multiplication by arbitrary elements of $R$ \\
                i.e. $x \in \ker f \implies rx, xr \in \ker f \ \forall r \in R$
        \end{itemize}

    \item \textbf{Lemma 3.4.5} \\
        Let $R$ and $S$ be rings and $f : R \to S$ a ring homomorphism.
        Then $\forall x,y, \in R$ and $m \in \mathbb{Z}$
        \begin{enumerate}
            \item $f(0_R)       = 0_S$
            \item $f(-x)        = -f(x)$
            \item $f(x-y)       = f(x) - f(y)$
            \item $f(m \cdot x) = m\cdot f(x)$
        \end{enumerate}
        Where $mx$ denotes the $m$-th multiple of $x$.

    \item \textbf{Definition 3.4.7} \emph{Ideal} \\
        A subset $I$ of a ring $R$ is an \emph{ideal}, written $I \trianglelefteq R$,
        if the following hold:
        \begin{enumerate}
            \item $I \neq \emptyset$
            \item $I$ is closed under subtraction (it's a subgroup)
            \item $\forall i \in I$ and $\forall r \in R$ we have $ri, ir \in I$
                ($I$ is closed under multiplication by elements of $R$)
        \end{enumerate}
        Ideals satisfy the properties of rings, except possibly the existence of a multiplicative
        identity.

        Ideals are subrings which are closed under multiplication with elements from the
        \emph{ring} --- not just elements from within the ideal!

    \item \textbf{Definition 3.4.11} \emph{Generated ideal} \\
        Let $R$ be a commutative ring and let $T \subset R$.
        Then the \emph{ideal of $R$ generated by $T$} is the set
        \[
            _R\langle T \rangle = \{ {r_1}{t_1} + \cdots + {r_m}{t_m} : t_1, \ldots, t_m \in T,
            r_1, \ldots, r_m \in R \}
        \]
        together with the zero element in the case $T = \emptyset$.

    \item \textbf{Proposition 3.4.14} \\
        Let $R$ be a commutative ring and let $T \subseteq R$.
        Then $_R \langle T \rangle$ is the smallest ideal of $R$ that contains $T$.

    \item \textbf{Definition 3.4.15} \emph{Principle ideal} \\
        Let $R$ be a commutative ring.
        An ideal $I \trianglelefteq R$ is called a \emph{principle ideal} if
        $I = \langle t \rangle$ for some $t \in R$.

    \item \textbf{Definition 3.4.17} \emph{Kernel} \\
        Let $R$ and $S$ be rings, and let $f : R \to S$ be a ring homomorphism.
        Since $F$ is in particular a group homomorphism from $(R,+)$ to $(S,+)$,
        the \emph{kernel} of $f$ already has a meaning:
        \[
            \ker f = \{ r \in R : f(r) = 0_S \}
        \]

    \item \textbf{Proposition 3.4.18} \\
        Let $R$ and $S$ be rings and $f : R \to S$ a ring homomorphism.
        Then $\ker f$ is an ideal of $R$.

    \item \textbf{Lemma 3.4.20} $f$ is injective if and only if $\ker f = \{0\}$

    \item \textbf{Lemma 3.4.21} The intersection of any collection of ideals of a ring $R$
        is an ideal of $R$.

    \item \textbf{Lemma 3.4.22} Let $I$ and $J$ be ideals of a ring $R$.
        Then
        \[
            I + J = \{a + b : a \in I, b \in J \}
        \]
        is an ideal of $R$.

    \item \textbf{Definition 3.4.23} \emph{Subring} \\
        Let $R$ be a ring.
        A subset $R' \subseteq R$ is a \emph{subring} of $R$ if $R'$ is itself a ring under the
        operations of addition and multiplication defined in $R$.

    \item \textbf{Proposition 3.4.26} Test for a subring \\
        Let $R$ be a ring, and $R' \subseteq R$.
        Then $R'$ is a subring if and only if
        \begin{enumerate}
            \item $R'$ has a multiplicative identity, and
            \item $R'$ is closed under subtraction, and
            \item $R'$ is closed under multiplication.
        \end{enumerate}

    \item \textbf{Proposition 3.4.29} Let $R$ and $S$ be rings and $f : R \to S$
        a ring homomorphism.
        \begin{enumerate}
            \item If $R'$ is a subring of $R$ then $f(R')$ is a subring of $S$.
                In particular, $\mathrm{f}$ is a subring of $S$.
            \item Assume that $f(1_R) = 1_S$.
                Then if $x$ is a unit in $R$, $f(x)$ is a unit is in $S$ and
                ${(f(x))}^{-1} = f{(x^{-1})}$.
                In this case $f$ restricts to a group homomorphism
                $f|_{R^\times} : R^\times \to S^\times$.
        \end{enumerate}
\end{itemize}

\subsection{Equivalence Relations}

\begin{itemize}

    \item \textbf{Definition 3.5.1} \emph{Relation} \\
        A \emph{relation} $R$ on a set $X$ is a subset $R \subseteq X \times X$.
        $R$ is an \emph{equivalence relation on $X$} when $\forall x,y,z \in X$ the following hold:
        \begin{enumerate}
            \item \emph{Reflexivity}: $xRx$
            \item \emph{Symmetry}: $xRy \iff yRx$
            \item \emph{Transitivity}: $xRy \ \text{and} \ yRz \implies xRz$
        \end{enumerate}

    \item \textbf{Definition 3.5.3} \\
        Suppose that $\sim$ is an equivalence relation on a set $X$.
        For $x \in X$ the set $E(x) \equiv \{z \in X : z\sim x\}$ is called the
        \emph{equivalence class} of $x$.

        A subset $E \subseteq X$ is called an \emph{equivalence class} for $\sim$ if
        $\exists x \in X \backepsilon E=E(x)$.

        An element of an equivalence class is called a \emph{representative} of the class.

        A subset $Z \subseteq X$ containing precisely one element from each equivalence class is
        called a \emph{system of representatives} for the equivalence relation.

    \item \textbf{Definition 3.5.5} \emph{Set of equivalence classes} \\
        Given an equivalence relation $\sim$ on the set $X$, the \emph{set of equivalence classes},
        which is a subset of $\mathcal{P}(X)$, is
        \[
            (X/\sim) \equiv \{E(x) : x \in X\}
        \]
        There is a canonical mapping $\mathrm{can} : X \to (X/\sim), \ x \mapsto E(x)$.
        It is obviously a surjection.

    \item \textbf{Remark} \\
        Suppose that $\sim$ is an equivalence relation on $X$.
        If $f : X \to Z$ is a mapping with the property that $x \sim y \implies f(x) = f(y)$,
        then there is a unique mapping $\overline{f} : (X \ \setminus \sim) \to Z$
        with $f = \overline{f} \circ \mathrm{can}$.
        Its definition is easy: $f(E(x)) = f(x)$.
        This property is called the \emph{universal property of the set of equivalence classes}.

    \item \textbf{Definition 3.5.7} \emph{Well-defined} \\
        $g : (X/\sim) \to Z$ is \emph{well-defined} if there is a mapping
        $f : X \to Z$ such that $f$ has the property
        $x \sim y \implies f(x) = f(y)$ and $g = \overline{f}$.
\end{itemize}

\subsection{Factor Rings and the First Isomorphic Theorem}

\begin{itemize}

    \item \textbf{Prelude} \\
        Let $f : R \to S$ be a ring homomorphism.
        \[
            x \sim y \iff f(x) = f(y) \iff f(x-y) = 0 \iff x - y \in \ker f
        \]
        Then:
        \[
            E(x) = x + \ker f \equiv \{x + k : k \in \ker f \}
        \]
        So we have that:
        \begin{itemize}
            \item the rule $x \sim y \iff x - y \in \ker f$ is an equivalence relation;
            \item the equivalence classes are the sets $x + \ker f$ for $x \in R$;
            \item the set of equivalence classes $(R \ / \sim)$ is a ring,
                isomorphic to a subring of $S$.
        \end{itemize}

    \item \textbf{Definition 3.6.1} \emph{Cosets} \\
        Let $I \trianglelefteq R$ be an ideal in a ring $R$.
        The set
        \[
            x+I \equiv \{x + i : i \in I\} \subseteq R
        \]
        is a \emph{coset of I in $R$}, or
        \emph{the coset of $x$ with respect to $I$ in $R$}.

    \item \textbf{Definition 3.6.3} \emph{Factor ring} \\
        Let $R$ be a ring, $I \trianglelefteq R$ be an ideal, and $\sim$ the equivalence relation
        defined by
        $x \sim y \iff x-y \in I$.
        Then $R/I$, the \emph{factor ring of $R$ by $I$} or the \emph{quotient of $R$ by $I$},
        is the set $(R \ / \sim)$ of cosets of $I$ in $R$.
        \[
            R/I = \{r + I : r \in R \}
        \]

    \item \textbf{Theorem 3.6.4} \\
        Let $R$ be a ring, and $I \trianglelefteq R$ an ideal.
        Then $R/I$ is a ring, where the operation of addition is defined by
        \[
            (x+I) \dot{+} (y+I) = (x+y) + I \quad \forall x,y \in R
        \]
        and multiplication is defined by
        \[
            (x+I) \cdot (y+I) = x y + I \quad \forall x,y \in R
        \]

    \item \textbf{Theorem 3.6.7} Universal Property of Factor Rings \\
        Let $R$ be a ring, and $I \trianglelefteq R$.
        \begin{enumerate}
            \item The mapping $\mathrm{can}: R \to R / I$ with $\mathrm{can}(r) = r + I$
                is a surjective ring homomorphism with kernel $I$.
            \item If $f : R \to S$ is a ring homomorphism with $f(I) = \{0_S\}$,
                so that $I \subseteq \ker f$, then there is a unique ring homomorphism
                $\overline{f}: R / I \to S$ such that $f = \overline{f} \circ \mathrm{can}$.
        \end{enumerate}

    \item \textbf{Theorem 3.6.9} First Isomorphic Theorem for Rings \\
        Let $R$ and $S$ be rings.
        Then every ring homomorphism $f: R \to S$ induces a ring isomorphism
        \[
            \overline{f} : R / \ker f \tilde{\to} \mathrm{im} f
        \]
\end{itemize}

\subsection{Modules}

\begin{itemize}
    \item \textbf{Definition 3.7.1}
        A \emph{(left) module $M$ over a ring $R$} is a pair consisting of an abelian group
        $M = (M, \dot{+})$ and a mapping
        \begin{align*}{}
            R \times M & \to M \\
            (r,a)      & \mapsto ra
        \end{align*}
        such that $\forall r,s \in R$ and $a,b \in M$ the following identities hold:
        \begin{align*}{}
            r(a \dot{+} b) & = (ra) \dot{+} (rb) & \text{(distributivity)} \\
            (r+s)a         & = (ra) \dot{+} (sa) & \text{(distributivity)}\\
            r(sa)          & = (rs)a             & \text{(associativity)}\\
            {1_R}a         & = a
        \end{align*}
        i.e.\ a vector space, but with a \emph{ring} instead of a \emph{field}.

    \item \textbf{Lemma 3.7.8} Let $R$ be a ring, and $M$ an $R$-module.
        \begin{enumerate}
            \item ${0_R}a = 0_M \ \forall a \in M$
            \item $r{0_M} = 0_M \ \forall r \in R$
            \item $(-r)a = r(-a) = -(ra), \quad \forall r \in R, a \in M$.
                (Here, the first negative is in $R$, and the last two negatives are in $M$.)
        \end{enumerate}

    \item \textbf{Definition 3.7.11} \\
        Let $R$ be a ring, and let $M,N$ be $R$-modules.
        A mapping $f : M \to N$ is an \emph{$R$-homomorphism} if the following hold
        $\forall a,b \in M$ and $r \in R$:
        \begin{align*}{}
            f(a+b) & = f(a) + f(b) \\
            f(ra)  & = rf(a)
        \end{align*}
        The \emph{kernel} of $f$ is $\ker f = \{a\in M : f(a) = 0_N \} \subseteq M$
        and the \emph{image} of $f$ is $\mathrm{im} f = \{f(a) : a \in M\} \subseteq N$. \\
        If $f$ is a bijection then it is an \emph{isomorphism}.

    \item \textbf{Definition 3.7.15} \\
        A non-empty subset $M'$ of an $R$-module $M$ is a \emph{submodule} if $M'$ is an $R$-module
        with respect to the operations of the $R$-module $M$ \emph{restricted} to $M'$.

    \item \textbf{Proposition 3.7.20} Test for a submodule \\
        Let $R$ be a ring and let $M$ be an $R$-module.
        A subset $M' \subseteq M$ is a submodule if and only if
        \begin{enumerate}
            \item $0_M \in M'$
            \item $a,b \in M' \implies a-b \in M'$
            \item $r \in R, a \in M' \implies ra \in M'$
        \end{enumerate}

    \item \textbf{Lemma 3.7.21} \\
        Let $f : M \to N$ be an $R$-homomorphism.
        Then $\ker f$ is a submodule of $M$ and $\mathrm{im} f$ is a submodule of $N$.

    \item \textbf{Lemma 3.7.22} \\
        Let $R$ be a ring, let $M$ and $N$ be $R$-modules and let $f : M \to N$ be an
        $R$-homomorphism.
        Then $f$ is injective if and only if $\ker f = \{0_M\}$.

    \item \textbf{Definition 3.7.23} \\
        Let $R$ be a ring, $M$ an $R$-module, and let $T \subseteq M$.
        Then the \emph{submodule of $M$ generated by $T$} is the set
        \[
            _R \langle T \rangle = \{{r_1}{t_1} + \cdots + {r_m}{t_m} : t_1, \ldots, t_m \in T,
            r_1, \ldots, r_m \in R \},
        \]
        together with the zero element in case $T = \emptyset$.\\
        The module $M$ is \emph{finitely generated} if it is generated by a finite set:
        $M = _r \langle \{ t_1, \ldots, t_n \}$. \\
        It is \emph{cyclic} f it is generated by a singleton:
        $M = _R \langle t \rangle$.

    \item \textbf{Lemma 3.7.28} Let $T \subseteq M$. Then $_r \langle T \rangle$
        is the smallest submodule of $M$ that contains $T$.

    \item \textbf{Lemma 3.7.29} The intersection of any collection of submodules of $M$ is a
        submodule of $M$.

    \item \textbf{Lemma 3.7.30} Let $M_1$ and $M_2$ be submodules of $M$.
        Then
        \[
            M_1 + M_2 = \{ a + b : a \in M_1, b \in M_2 \}
        \]
        is a submodule of $M$.

    \item \textbf{Definition 3.7.31.1} \emph{Coset} \\
        Let $R$ be a ring, $M$ an $R$-module, and $N$ a submodule of $M$.
        For each $a \in M$, the \emph{coset of $a$ with respect to $N$ in $M$} is
        \[
            a + N = \{ a + b : b \in N \}.
        \]
        It is a coset of $N$ in the abelian group $M$ and is is an equivalence class for the
        equivalence relation $a \sim b \iff a-b \in N$.

    \item \textbf{Definition 3.7.31.2} \emph{Factor} \\
        $M/N$, the \emph{factor of $M$ by $N$} or the \emph{quotient of $M$ by $N$},
        is the set $(M \ / \sim)$ of all cosets of $N$ in $M$.
        \[
            M/N = \{ a + N : a \in M \}
        \]
        This becomes an $R$-module by introducing the operations of addition and multiplication
        as follows:
        \begin{align*}{}
            (a + N) \dot{+} (b + N) & = (a + b) + N \\
            r(a + N)                & = ra + N
        \end{align*}
        for all $a, b \in M, r \in R$.

    \item \textbf{Theorem 3.7.31.3} \emph{Factor module}
        \begin{itemize}
            \item The zero of $M/N$ is the coset $0_{M/N} = 0_M + N$.
            \item The negative of $a + N \in M/N$ is the coset $-(a+N) = (-a)+N$.
            \item The $R$-module $M/N$ is the \emph{factor module} of $M$ by the submodule $N$.
        \end{itemize}

    \item \textbf{Theorem 3.7.32} The Universal Property of Factor Modules \\
        Let $R$ be a ring, and let $L$ and $M$ be $R$-modules, and $N$ a sub-module of $M$.
        \begin{enumerate}
            \item The mapping $\mathrm{can} : M \to M/N$ sending $a$ to $a+N, \ \forall a \in M$
                is a surjective $R$-homomorphism with kernel $N$.
            \item If $f : M \to L$ is an $R$-homomorphism with $f(N) = \{0_L\}$,
                so that $N \subseteq \ker f$,
                then there is a unique homomorphism $\overline{f} : M/N \to L$
                such that $f = \overline{f} \circ \mathrm{can}$.
        \end{enumerate}

    \item \textbf{Theorem 3.7.33} First Isomorphism Theorem for Modules \\
        Let $R$ be a ring and let $M$ and $N$ be $R$-modules.
        Then every $R$-homomorphism $f : M \to N$ induces a $R$-isomorphism
        \[
            \overline{f} : M / \ker f \to \mathrm{im} f
        \]
\end{itemize}

\section{Determinants \& Eigenvalues Redux}

\subsection{The sign of a permutation}

\begin{itemize}
    \item \textbf{Definition 4.1.1} \emph{Transposition} \\
        The group of all permutations of the set $\{1,2,\ldots,n\}$, also known as bijections from
        $\{1,2,\ldots,n\}$ to itself, is denoted by $\mathfrak{S}_n$ and called the
        \emph{$n$-th symmetric group}.
        It is a group under composition and has $n$! elements.

        A \emph{transposition} is a permutation that swaps two elements of the set
        and leaves all the others unchanged.

    \item \textbf{Definition 4.1.2} \emph{Inversion \& Sign} \\
        An \emph{inversion} of a permutation $\sigma \in \mathfrak{S}_n$ is a pair $(i,j)$
        such that $1 \leq i < j \leq n$ and $\sigma(i) > \sigma(j)$.
        The number of inversions of the permutation $\sigma$ is called the \emph{length of}
        $\sigma$ and written $\ell(\sigma)$.
        In formulas:
        \[
            \ell(\sigma) = |\{(i,j) : i < j \ \mathrm{but} \ \sigma(i) > \sigma(j) \}|
        \]
        The \emph{sign of} $\sigma$ is defined to be the parity of the number of inversions of
        $\sigma$.
        In formulas:
        \[
            \mathrm{sgn}(\sigma) = {(-1)}^{\ell(\sigma)}
        \]
        A permutation whose sign is $+1$, in other words which has even length, is called an
        \emph{even permutation},
        while a permutation whose sign is $-1$, in other words which has odd length,
        is called an \emph{odd permutation}.

    \item \textbf{Lemma 4.1.5} (Multiplicativity of the sign) \\
        For each $n \in \mathbb{N}$ the sign of a permutation produces a group homomorphism
        $\mathrm{sgn} \ : \ \mathfrak{S}_n \to \{+1, -1\}$
        from the symmetric group to the two-element group of signs.
        In formulas:
        \[
            \mathrm{sgn}(\sigma \tau) = \mathrm{sgn}(\sigma) \, \mathrm{sgn}(\tau) \quad
            \forall \sigma, \tau \in \mathfrak{S}_n
        \]

    \item \textbf{Definition 4.1.7} \emph{Alternating group} \\
        For $n \in \mathbb{N}$, the set of even permutations in $\mathfrak{S}_n$ forms a subgroup
        of $\mathfrak{S}_n$ because it is the kernel of the group homomorphism
        $\mathrm{sgn} : \mathfrak{S}_n \to \{+1, -1\}$.
        This group is the \emph{alternating group} and is denoted $A_n$.

\end{itemize}

\subsection{Determinants \& what they mean}

\begin{itemize}
    \item \textbf{Definition 4.2.1}
        Let $R$ be a commutative ring and $n \in \mathbb{N}$. \\
        The \emph{determinant} is a mapping
        $\det : \mathrm{Mat}(n;R) \to R$ from square matrices with coefficients in $R$
        to the ring $R$ that is given by the following formula:
        \[
            A =
            \begin{bmatrix}
                a_{11} & \cdots & a_{1n} \\
                \vdots & \ddots & \vdots \\
                a_{n1} & \cdots & a_{nn} \\
            \end{bmatrix}
            \mapsto \det (A) =
            \sum_{\sigma \in \mathfrak{S}_n}
            \mathrm{sgn}(\sigma) a_{1\sigma(1)} \ldots a_{n\sigma(n)}
        \]
        This formula is called the \emph{Leibnitz formula}. \\
        The degenerate case $n=0$ assigns the value $1$ as the determinant of the ``empty matrix''.

    \item \emph{The connection between determinants and volumes} \\
        The determinant of a matrix is equal to the scaling factor it performs.

    \item \emph{The connection between determinants and orientation} \\
        The sign of the determinant determines the orientation:
        $\det = +1$ preserves the orientation;
        $\det = -1$ reverses the orientation.

\end{itemize}

\subsection{Characterising the determininant}

\begin{itemize}
    \item \textbf{Definition 4.3.1} \emph{Bi-linear forms} \\
        Let $U,V, W$ be $F$-vector spaces. \\
        A \emph{bi-linear form on $U \times V$ with values in $W$} is a mapping
        $H:U \times V \to W$ which is a linear mapping in both of its entries. \\
        This means that it must satisfy the following properties for all
        $u_1, u_2 \in U; \ v_1, v_2 \in V; \ \lambda \in F$:
        \begin{align*}{}
            H(u_1 + u_2, v_1)   &= H(u_1, v_1) + H(u_2, v_1) \\
            H(u_1, v_1 + v_2)   &= H(u_1, v_1) + H(u_1, v_2) \\
            H(u_1, \lambda v_1) &= \lambda H(u_1, v_1) \\
            H(\lambda u_1, v_1) &= \lambda H(u_1, v_1)
        \end{align*}
        The first two conditions state that for any fixed $v \in V$ the mapping
        $H(-, v) : U \to W$ is linear.
        $H$ is a \emph{bi-linear form}.
        A bi-linear form $H$ is \emph{symmetric} if $U=V$ and
        \[
            H(u, v) = H(v, u) \quad \forall u, v \in U
        \]
        while it is \emph{alternating} or \emph{antisymmetric} if $U=V$ and
        \[
            H(u, u) = 0 \quad \forall u \in U
        \]

    \item \textbf{Definition 4.3.3} \emph{Multi-linear forms} \\
        Let $V_1, \ldots, V_n, W$ be $F$-vector spaces.
        A mapping $H : V_1 \times V_2 \times \cdots \times V_n \to W$ is a
        \emph{multi-linear form} or \emph{multi-linear} if for each $j$,
        the mapping $V_j \to W$ defined by $v_j \mapsto H(v_1, \ldots, v_j, \ldots, v_n)$,
        with $v_i \in V_i$ arbitrary fixed vectors of $V_i$ for $i \neq j$, is linear.
        In the case $n=2$, this is exactly the definition of a bi-linear mapping.

    \item \textbf{Definition 4.3.4} \emph{Alternating} \\
        Let $V$ and $W$ be $F$-vector spaces.
        A multi-linear form $H : V \times \cdots \times V \to W$
        is \emph{alternating} if it vanishes on every $n$-tuple of elements of $V$ that has at
        least two entries equal, in other words if:
        \[
            (\exists i \neq j \ \mathrm{with} \ v_i = v_j) \implies
            H(v_1, \ldots, v_i, \ldots, v_j, \ldots, v_n) = 0
        \]
        In the case $n=2$, this is exactly the definition of an alternating or anti-symmetric
        bi-linear mapping.

    \item \textbf{Theorem 4.3.6} Characterisation of the determinant \\
        Let $F$ be a field.
        The mapping
        \[
            \det : \mathrm{Mat}(n;F) \to F
        \]
        is the unique alternating multi-linear form on $n$-tuples of column vectors
        with values in $F$ that takes the value $1_F$ on the identity matrix.

\end{itemize}

\subsection{Rules for calculating with determinants}

\begin{itemize}
    \item \textbf{Theorem 4.4.1} Multiplicativity of the determinant \\
        Let $R$ be a commutative ring and let $A, B \in \mathrm{Mat}(n; R)$.
        Then
        \[
            \det(A B) = \det(A)\det(B)
        \]

    \item \textbf{Theorem 4.4.2} Determinantal criterion for invertibility \\
        The determinant of a square matrix with entries in a field $F$ is non-zero
        if and only if the matrix is invertible.

    \item \textbf{Definition 4.4.6} \emph{Cofactor} \\
        Let $A \in \mathrm{Mat}(n; R)$ for some commutative ring $R$ and $n \in \mathbb{N}$.
        Let $i, j \in (1,n) \subset \mathbb{N}$.
        Then the $(i, j)$ \emph{cofactor of} $A$ is
        $C_{ij} = {(-1)}^{i+j} \det(A \langle i, j \rangle)$
        where $A \langle i, j \rangle$ is the matrix obtained by deleting the $i$-th row and the
        $j$-th column.

    \item \textbf{Theorem 4.4.7} Laplace's expansion of the determinant \\
        Let $A = (a_{ij})$ be an $(n \times n)$ matrix with entries from a commutative ring $R$.

        For a fixed $i$, the \emph{$i$-th row expansion of the determinant} is
        \[
            \det(A) = \sum_{j=1}^n a_{ij} C_{ij}
        \]
        and for a fixed $j$, the \emph{$j$-th column expansion of the determinant} is
        \[
            \det(A) = \sum_{i=1}^n a_{ij} C_{ij}
        \]

    \item \textbf{Definition 4.4.8} \emph{Adjugate matrix} \\
        Let $A$ be an $(n \times n)$ matrix whose entries are
        $\mathrm{adj}{
        (A)}_{ij} = C_{ji}$ where $C_{ji}$ is the $(j, i)$ cofactor.

    \item \textbf{Theorem 4.4.9} Cramer's rule \\
        Let $A$ be an $(n \times n)$ matrix with entries in a commutative ring $R$.
        Then
        \[
            A \cdot \mathrm{adj}(A) = (\det A)I_n
        \]

    \item \textbf{Corollary 4.4.11} Invertibility of matrices \\
        A square matrix with entries in a commutative ring $R$ is invertible if and only if its
        determinant is a unit in $R$.
        That is, $A \in \mathrm{Mat}(n; R)$ is invertible if and only if $\det(A) \in R^\times$.

\end{itemize}

\subsection{Eigenvalues \& Eigenvectors}

\begin{itemize}
    \item \textbf{Definition 4.5.1} \emph{Eigenvalue} \\
        Let $f : V \to V$ be an endomorphism of an $F$-vector space $V$.
        A scalar $\lambda \in F$ is an \emph{eigenvalue} of $f$ if and only if there exists
        a non-zero vector $\textbf{v} \in V$ such that $f(\textbf{v}) = \lambda \textbf{v}$.

        Each such vector is called an \emph{eigenvector of $f$ with eigenvalue $\lambda$}.

        For any $\lambda \in F$, the \emph{eigenspace of $f$ with eigenvalue $\lambda$} is
        \[
            E(\lambda, f) = \{ \textbf{v} \in V \ : \ f(\textbf{v}) = \lambda \textbf{v} \}
        \]

    \item \textbf{Theorem 4.5.4} Existence of Eigenvalues \\
        Each endomorphism of a non-zero finite-dimensional vector space over an algebraically
        closed field has an eigenvalue.

    \item \textbf{Definition 4.5.6} \emph{Characteristic polynomial} \\
        Let $R$ be a commutative ring and let $A \in \mathrm{Mat}(n; R)$ be a square matrix with
        entries in $R$.
        The polynomial $\det (A - x I_n) \in R[x]$ is called the
        \emph{characteristic polynomial of the matrix $A$}.
        It is denoted by
        \[
            \chi_A (x) \equiv \det(A - x I_n)
        \]
        where $\chi$ stands for $\chi$aracteristic.

    \item \textbf{Theorem 4.5.8} Eigenvalues and characteristic polynomials \\
        Let $F$ be a field and $A \in \mathrm{Mat}(n; F)$ a square matrix with entries in $F$.
        The eigenvalues of the linear mapping $A : F^n \to F^n$ are exactly the roots of the
        characteristic polynomial $\chi_A$.

\end{itemize}

\subsection{Triangularisable, Diagonalisable, \& the Cayley-Hamilton theorem}

\begin{itemize}
    \item \textbf{Proposition 4.6.1} Triangularisability \\
        Let $f : V \to V$ be an endomorphism of a finite-dimensional $F$-vector space $V$.
        The following two statements are equivalent:
        \begin{enumerate}
            \item The vector space $V$ has an ordered basis
                $\mathcal{B} = (\textbf{v}_1, \textbf{v}_2, \ldots, \textbf{v}_n)$ such that
                \begin{align*}{}
                    f(\textbf{v}_1) &= a_{11}\textbf{v}_1 \\
                    f(\textbf{v}_2) &= a_{12}\textbf{v}_1 + a_{22}\textbf{v}_2 \\
                                    &\vdots \\
                    f(\textbf{v}_n) &= a_{1n}\textbf{v}_1 + a_{2n}\textbf{v}_2 + \cdots +
                    a_{nn}\textbf{v}_n \in V
                \end{align*}
                (so that the first basis vector $\textbf{v}_1$ is an eigenvector,
                with eigenvalue $a_{11}$) or equivalently such that the $n \times n$ matrix
                $_\mathcal{B}{[f]}_\mathcal{B} = (a_{ij})$ representing $f$ with respect to
                $\mathcal{B}$ is upper triangular.

                \[
                    A =
                    \begin{bmatrix}{}
                        a_{11} & a_{12} & a_{13} & \cdots & a_{1n} \\
                        0      & a_{22} & a_{23} & \cdots & a_{1n} \\
                        0      & 0      & a_{33} & \cdots & a_{1n} \\
                        \vdots & \vdots & \vdots & \ddots & \vdots \\
                        0      & 0      & 0      & \cdots & a_{nn}
                    \end{bmatrix}
                \]
                When this happens, $f$ is \emph{triangularisable}.

            \item The characteristic polynomial $\chi_{f(x)}$ of $f$ decomposes into linear
                factors in $F[x]$.

        \end{enumerate}

    \item \textbf{Definition 4.6.5} \emph{Diagonalisable} \\
        An endomorphism $f : V \to V$ of an $F$-vector space $V$ is \emph{diagonalisable}
        if and only if there exists a basis of $V$ consisting of eigenvectors of $f$.

        If $V$ is finite-dimensional, then this is the same as saying that there exists an
        ordered basis $\mathcal{B} = \{ \textbf{v}_1, \ldots, \textbf{v}_n \}$
        such that the corresponding matrix representing $f$ is diagonal,
        that is $_\mathcal{B}{[f]}_\mathcal{B} = \mathrm{diag}(\lambda_1, \ldots, \lambda_n$).
        In this case, of course, $f(\textbf{v}_i) = \lambda_i v_i$.

        A square matrix $A \in \mathrm{Mat}(n; F)$ is \emph{diagonalisable} if and only if
        the corresponding linear mapping $F^n \to F^n$ given by the left multiplication of $A$
        is diagonalisable.
        This just means that $A$ is conjugate to a diagonal matrix:
        there exists an invertible matrix $P \in \mathrm{GL}(n; F)$ such that
        $P^{-1} A P = \mathrm{diag}(\lambda_1, \ldots, \lambda_n)$.
        In this case, the columns of $P$ are the vectors of a basis of $F^n$ consisting of
        eigenvectors of $A$ with eigenvalues $\lambda_1, \ldots, \lambda_n$.

    \item \textbf{Lemma 4.6.8} Linear independence of Eigenvectors \\
        Let $f : V \to V$ be an endomorphism of a vector space $V$ and let
        $\textbf{v}_1, \ldots, \textbf{v}_n$ be eigenvectors of $f$ with pairwise different
        eigenvalues $\lambda_1, \ldots, \lambda_n$. \\
        Then the vectors $\textbf{v}_1, \ldots, \textbf{v}_n$ are linearly independent.
    
    \item \textbf{Theorem 4.6.9} Cayley-Hamilton Theorem \\
        Let $A \in \mathrm{Mat}(n; R)$ be a square matrix with entries in a commutative ring $R$.
        Then evaluating its characteristic polynomial $\chi_A(x) \in R[x]$ at the matrix $A$ gives
        zero.

\end{itemize}

\subsection{Google's PageRank Algorithm}

\section{Reference}

\subsection{Terminology of Algebraic Structures}

\begin{tabular}{cccc}
    \toprule
               & \emph{Associativity} & \emph{Identity} & \emph{Inverses} \\
    \midrule
    Group      & Yes                  & Yes             & Yes \\
    Monoid     & Yes                  & Yes             & No  \\
    Semi-group & Yes                  & No              & No  \\
    Magma      & No                   & No              & No  \\
    \bottomrule
\end{tabular}

Ring  = (Group, Monoid)

Field = (Group, Group)

\end{document}
